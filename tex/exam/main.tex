\documentclass[10pt]{article}

\usepackage[utf8]{inputenc}
\usepackage[T1]{fontenc}
\usepackage[polish]{babel}
\usepackage{listings}
\usepackage{xcolor}
\usepackage{amsmath, amssymb, amsthm}
\usepackage{graphicx}
\usepackage{geometry}
\usepackage{hyperref}
\usepackage{fancyhdr}
\usepackage{enumitem}
\usepackage{titlesec}

\geometry{
    a4paper,
    left=25mm,
    right=25mm,
    top=25mm,
    bottom=25mm
}

\pagestyle{fancy}
\fancyhf{}
\fancyhead[L]{PW}
\fancyhead[R]{\thepage}
\fancyfoot[C]{}

\newtheorem{theorem}{Theorem}[section]
\newtheorem{lemma}[theorem]{Lemma}
\newtheorem{corollary}[theorem]{Corollary}
\newtheorem{definition}[theorem]{Definition}
\newtheorem{proposition}[theorem]{Proposition}
\newtheorem{remark}[theorem]{Remark}

\newcommand{\R}{\mathbb{R}}
\newcommand{\C}{\mathbb{C}}
\newcommand{\N}{\mathbb{N}}
\newcommand{\Z}{\mathbb{Z}}
\newcommand{\Q}{\mathbb{Q}}
\newcommand{\eps}{\varepsilon}
\newcommand{\norm}[1]{\left\lVert#1\right\rVert}
\newcommand{\abs}[1]{\left|#1\right|}

\titleformat{\section}{\normalfont\Large\bfseries}{\thesection}{1em}{}
\titleformat{\subsection}{\normalfont\large\bfseries}{\thesubsection}{1em}{}
\titleformat{\subsubsection}{\normalfont\normalsize\bfseries}{\thesubsubsection}{1em}{}

% Definicja stylu dla JSON
\lstdefinelanguage{json}{
    basicstyle=\ttfamily\footnotesize, % Ustawienie czcionki stałej szerokości
    showstringspaces=false,
    breaklines=true,
    frame=single,
    literate=
     *{0}{{{\color{numb}0}}}{1}
      {1}{{{\color{numb}1}}}{1}
      {2}{{{\color{numb}2}}}{1}
      {3}{{{\color{numb}3}}}{1}
      {4}{{{\color{numb}4}}}{1}
      {5}{{{\color{numb}5}}}{1}
      {6}{{{\color{numb}6}}}{1}
      {7}{{{\color{numb}7}}}{1}
      {8}{{{\color{numb}8}}}{1}
      {9}{{{\color{numb}9}}}{1}
      {:}{{{\color{punct}{:}}}}{1}
      {,}{{{\color{punct}{,}}}}{1}
      {\{}{{{\color{delim}{\{}}}}{1}
      {\}}{{{\color{delim}{\}}}}}{1}
      {[}{{{\color{delim}{[}}}}{1}
      {]}{{{\color{delim}{]}}}}{1},
}
\colorlet{punct}{red!60!black}
\colorlet{numb}{magenta!60!black}
\colorlet{delim}{blue!60!black}

\title{Egzamin końcowy.}
\author{Łukasz Ambroziak}

\begin{document}

\maketitle

\section{Pytania}

\subsection{Czym jest, do czego służy i jak działa Apache Cassandra?}

Apache Cassandra to rozproszona baza danych NoSQL, która została zaprojektowana do obsługi dużych ilości danych rozproszonych na wielu serwerach bez pojedynczego punktu awarii. Oto szczegółowy opis jej cech i funkcji:

Czym jest Apache Cassandra?

Apache Cassandra to wysoce skalowalna i dostępna baza danych typu NoSQL, która pierwotnie została opracowana przez Facebooka a następnie przekazana do społeczności open-source jako projekt Apache. Została stworzona z myślą o obsłudze dużych wolumenów danych i zapewnieniu wysokiej dostępności, nawet w przypadku awarii niektórych węzłów w klastrze.

Do czego służy Apache Cassandra?

\begin{enumerate}
  \item Obsługa dużych zbiorów danych: Cassandra jest idealna do przechowywania i zarządzania ogromnymi ilościami danych, które mogą być rozproszone na wielu serwerach.
  \item Wysoka dostępność i niezawodność: Dzięki replikacji danych i mechanizmom failover, Cassandra zapewnia ciągłość działania nawet w przypadku awarii poszczególnych węzłów.
  \item Elastyczne skalowanie: System można łatwo skalować poziomo, dodając nowe węzły do klastra bez przestojów.
  \item Rozproszone architektury: Jest często wykorzystywana w systemach rozproszonych, takich jak globalne aplikacje webowe, systemy analityczne, platformy e-commerce czy sieci społecznościowe.
\end{enumerate}

Jak działa Apache Cassandra?

\begin{enumerate}
  \item Architektura rozproszona: Cassandra działa w oparciu o architekturę peer-to-peer, gdzie każdy węzeł w klastrze ma taką samą rolę. Nie ma pojedynczego węzła master, co eliminuje pojedyncze punkty awarii.
  \item Replikacja danych: Dane w Cassandra są replikowane na wiele węzłów, co zapewnia redundancję i wysoką dostępność. Liczba replik jest konfigurowalna.
  \item Model danych: Cassandra używa modelu kolumnowego, gdzie dane są przechowywane w tabelach składających się z wierszy i kolumn. Każda tabela ma klucz podstawowy, który identyfikuje unikalnie każdy wiersz.
\end{enumerate}

\end{document}
